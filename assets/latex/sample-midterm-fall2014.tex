
\documentclass[11pt]{article}
\usepackage{txfonts}
\usepackage{graphicx}
\usepackage{mygb4e}
\usepackage{solution}

\hidesolutions
\setlength\oddsidemargin{0.01in}
\setlength\topmargin{-1in}
\setlength\textwidth{6.9in}
\setlength\textheight{9.5in} 

\newcommand{\nl}{\mbox{$\langle cr \rangle$}}

\raggedright

\newcommand{\valueof}[1]{\lbrack\!\lbrack #1 \rbrack\!\rbrack}
\newcommand{\refp}[1]{(\protect\ref{#1})}
\def\rwd#1 {\mbox{#1}}
\def\N{\ensuremath{\mathcal{N}}}
\def\implies{\ensuremath{\Rightarrow}}
% Set some text inside an fbox the full width of the line, with the frame
% sticking out into the margin.
\long\def\framepar#1{\par\noindent\hbox to \textwidth {\hskip-\fboxsep
\fbox{\parbox{\the\textwidth}{#1}}}}

\begin{document}
%\setlength{\baselineskip}{12pt}

\begin{center}
{\Large\bf
Sample Midterm: Natural Language Processing
}
\end{center}

\begin{exe}

\ex \textbf{Language Modeling}

Consider a language model over character sequences that computes the probability of
a word based on the characters in that word, so if word $w = c_0, c_1, \ldots, c_n$ then
$P(w) = P(c_0, \ldots, c_n)$. Let us assume that the language model is defined as
a bigram character model $P(c_i \mid c_{i-1})$ where 
\begin{eqnarray*} 
P(c_0, \ldots, c_n) = \prod_{i=1,2,\ldots,n} P(c_i \mid c_{i-1}) \label{eqn:bigram} 
\end{eqnarray*}
%For convenience we assume that we have explicit word boundaries: $c_0 = \texttt{b}$ and $c_n = \texttt{e}$ where \texttt{b} stands for \textit{begin marker} and \texttt{e} stands for \textit{end marker}.
Katz backoff smoothing is defined as follows:
\[ P_{\textit{katz}}(c_i~\mid~c_{i-1}) = \left\{
\begin{array}{cl}
\frac{ r^\ast(c_{i-1}, c_i) }{ r(c_{i-1}) } & \textrm{ if $r(c_{i-1}, c_i) > 0$} \\
\alpha(c_{i-1}) P_{\textit{katz}} (c_i) & \textrm{otherwise}
\end{array}
\right. \]
where $r(\cdot)$ provides the (unsmoothed) frequency from training data and $r^\ast(\cdot)$ is the Good-Turing estimate of the frequency $r$.
%
%defined as follows:
%\[ r^\ast(c_{i-1}, c_i) = (r(c_{i-1}, c_i)+1) \times \frac{n_{r(c_{i-1}, c_i)+1}}{n_{r(c_{i-1}, c_i)}} \]
%
%where $n_{r(c_{i-1}, c_i)}$ is the number of
%different $c_{i-1}, c_i$ types observed with count $r(c_{i-1}, c_i)$. We assume that linear interpolation has provided all missing $n_{r(\cdot)}$ values required.

%which is defined as $r^\ast = (r+1) \frac{n_{r+1}}{n_r}$
%, where $n_r$ is the number of types with frequency $r$.
Provide the equation for $\alpha(c_{i-1})$ that ensures that $P_{\textit{katz}}(c_i~\mid~c_{i-1})$ is a proper probability.

\begin{soln}

Step by step derivation below. We are just looking for the end result.

\begin{eqnarray*}
\sum_{c_i} \left( \frac{r^\ast(c_{i-1}, c_i)}{r(c_{i-1})} + \alpha(c_{i-1}) P_\textit{katz}(c_i) \right) &=& 1 \\
\sum_{c_i : r(c_{i-1}, c_i) > 0} \frac{r^\ast(c_{i-1}, c_i)}{r(c_{i-1})} + \alpha(c_{i-1}) \sum_{c_i : r(c_{i-1}, c_i) = 0} P_\textit{katz}(c_i) &=& 1 \\
\alpha(c_{i-1}) \sum_{c_i : r(c_{i-1}, c_i) = 0} P_\textit{katz}(c_i) &=& 1 - \left( \sum_{c_i : r(c_{i-1}, c_i) > 0} \frac{r^\ast(c_{i-1}, c_i)}{r(c_{i-1})}  \right) \\
\alpha(c_{i-1}) &=& \frac{1 - \left( \sum_{c_i : r(c_{i-1}, c_i) > 0} \frac{r^\ast(c_{i-1}, c_i)}{r(c_{i-1})}  \right)}{\sum_{c_i : r(c_{i-1}, c_i) = 0} P_\textit{katz}(c_i)} \\
\alpha(c_{i-1}) &=& \frac{1 - \left( \sum_{c_i : r(c_{i-1}, c_i) > 0} \frac{r^\ast(c_{i-1}, c_i)}{r(c_{i-1})}  \right)}{1 - \left( \sum_{c_i : r(c_{i-1}, c_i) > 0} P_\textit{katz}(c_i) \right) } 
\end{eqnarray*}

Also acceptable is the somewhat less precise answer which assumes $\sum_{c_i} P_\textit{katz}(c_i) = 1$:
\[ \alpha(c_{i-1}) = 1 - \sum_{c_i} \frac{ r^\ast(c_{i-1}, c_i) }{ r(c_{i-1}) } \]

\end{soln}

\bigskip
\ex\label{hmm} \textbf{Hidden Markov Models}:

The probability model $P(t_i \mid t_{i-2}, t_{i-1})$ 
is provided below 
where each $t_i$ is a part of speech tag, e.g. the sixth row of
the left table below corresponds to
$P(\textrm{D} \mid \textrm{N}, \textrm{V}) = \frac{1}{3}$. 
Also provided is $P(w_i \mid t_i)$ that a word $w_i$ has a part of speech 
tag $t_i$, e.g. the seventh line of the right table below corresponds to $P(\textrm{flies} \mid \textrm{V}) = \frac{1}{2}$. 

\[ \begin{array}{ccc}
\begin{tabular}{|c|c|c|c|}
\hline
$P(t_i \mid t_{i-2}, t_{i-1})$ & $t_{i-2}$ & $t_{i-1}$ & $t_i$ \\
\hline \hline
1 & \texttt{bos} & \texttt{bos} & N \\
\hline
$\frac{1}{2}$ & \texttt{bos} & N & N \\
\hline
$\frac{1}{2}$ & \texttt{bos} & N & V \\
\hline
$\frac{1}{2}$ & N & N & V \\
\hline
$\frac{1}{2}$ & N & N & P \\
\hline
$\frac{1}{3}$ & N & V & D \\
\hline
$\frac{1}{3}$ & N & V & V \\
\hline
$\frac{1}{3}$ & N & V & P \\
\hline
1 & V & D & N \\
\hline
1 & V & V & D \\
\hline
1 & N & P & D \\
\hline
1 & V & P & D \\
\hline
1 & P & D & N \\
\hline
1 & D & N & \texttt{eos} \\
\hline
\end{tabular}
&&
\begin{tabular}{|c|c|c|}
\hline
$P(w_i \mid t_i)$ & $t_i$ & $w_i$ \\
\hline\hline
1 & D & an \\
\hline
$\frac{2}{5}$ & N & time \\
\hline
$\frac{2}{5}$ & N & arrow \\
\hline
$\frac{1}{5}$ & N & flies \\
\hline
1 & P & like \\
\hline
$\frac{1}{2}$ & V & like \\
\hline
$\frac{1}{2}$ & V & flies \\
\hline
1 & \texttt{eos} & \texttt{eos} \\
\hline
1 & \texttt{bos} & \texttt{bos} \\
\hline
\end{tabular}
\end{array}
\]

The part of speech tag definitions are:
\texttt{bos} ({\em begin sentence marker}), N ({\em noun}), V ({\em
  verb}), D ({\em determiner}), P ({\em preposition}), \texttt{eos}
({\em end of sentence marker}).

\smallskip

\begin{xlist}
  
  \ex{\label{tri2} Provide a Hidden Markov Model ($hmm$) that uses the trigram
  part of speech probability $P(t_i \mid t_{i-2}, t_{i-1})$ 
  as the transition probability $P_{hmm}(s_j \mid s_k)$ and the
  probability $P(w_i \mid t_i)$ as the emission probability $P_{hmm}(w_j \mid s_j)$. 

   {\bf Important:} Provide the $hmm$ in the form of two tables as shown below.
   The first table contains transitions between states in the $hmm$ and the 
   transition probabilities and the second table contains the words emitted
   at each state and the emission probabilities. Do not provide entries with
   zero probability.

\begin{tabular}{|l|l|l|}
\hline
from-state $s_k$ & to-state $s_j$ & $P(s_j \mid s_k)$ \\
\hline
& & \\
& & \\
\end{tabular}
\hspace{.5in}
\begin{tabular}{|l|l|l|}
\hline
state $s_j$ & emission $w$ & $P(w \mid s_j)$ \\
\hline
 &  &  \\
 &  &  \\
% $bos,bos$ & $bos$ & 1 \\
% $bos,N$   & time & $\frac{2}{5}$ \\
\end{tabular}

   {\bf Hint:} In your $hmm$ the state $\langle N, \texttt{eos} \rangle$
   will have emission of word \texttt{eos} with probability 1
   and will not have transitions to any other states.

%  {\em Hint:} Create states in the $hmm$ of the type $\langle t_a, t_b \rangle$
%  such that a transition probability from $\langle t_a, t_b \rangle$ to 
%  $\langle t_b, t_c \rangle$ is the trigram probability $P(t_c \mid t_a, t_b)$
%  and emission probability of $w$ from state $\langle t_a, t_b \rangle$ is
%  $P(w \mid t_b)$.
   
   \begin{soln}
   Here are the two tables that define the HMM, the transition table on the left and the emission table on the right:
   
\begin{tabular}{|l|l|ll|}
\hline
from-state $s_k$ & to-state $s_j$ & $P(s_j \mid s_k)$ & \\
\hline
 $bos,bos$ & $bos,N$ & $P(N \mid bos,bos)$& $1$ \\ \hline
 $bos,N$   & $N,N$   & $P(N \mid bos,N)$& $\frac{1}{2}$ \\
 $bos,N$   & $N,V$   & $P(V \mid bos,N)$& $\frac{1}{2}$ \\ \hline
 $N,N$     & $N,V$   & $P(V \mid N,N)$ & $\frac{1}{2}$ \\
 $N,N$     & $N,P$   & $P(P \mid N,N) $& $\frac{1}{2}$ \\ \hline
 $N,V$     & $V,D$   & $P(D \mid N,V)$ & $\frac{1}{3}$ \\
 $N,V$     & $V,V$   & $P(V \mid N,V) $& $\frac{1}{3}$ \\
 $N,V$     & $V,P$   & $P(P \mid N,V) $& $\frac{1}{3}$ \\ \hline
 $V,D$     & $D,N$   & $P(N \mid V,D)$& $1$ \\ \hline 
 $V,V$     & $V,D$   & $P(D \mid V,V)$& $1$ \\ \hline 
 $N,P$     & $P,D$   & $P(D \mid N,P)$& $1$ \\ \hline 
 $V,P$     & $P,D$   & $P(D \mid V,P)$& $1$ \\ \hline 
 $P,D$     & $D,N$   & $P(N \mid P,D)$& $1$ \\ \hline 
 $D,N$     & $N,eos$ & $P(eos \mid D,N)$& $1$ \\
 \hline
\end{tabular}
\hspace{.5in}
\begin{tabular}{|l|l|l|}
\hline
state $s_j$ & emission $w$ & $P(w \mid s_j)$ \\
\hline
 $bos,bos$ & $bos$ & 1 \\ \hline
 $bos,N$   & time & $\frac{2}{5}$ \\
 $bos,N$     & arrow & $\frac{2}{5}$ \\
 $bos,N$     & flies & $\frac{1}{5}$ \\ \hline
 $N,N$     & time & $\frac{2}{5}$ \\
 $N,N$     & arrow & $\frac{2}{5}$ \\
 $N,N$     & flies & $\frac{1}{5}$ \\ \hline
 $N,V$     & like & $\frac{1}{2}$ \\
 $N,V$     & flies & $\frac{1}{2}$ \\ \hline
 $V,D$     & an & $1$ \\ \hline
 $V,V$     & like & $\frac{1}{2}$ \\
 $V,V$     & flies & $\frac{1}{2}$ \\ \hline
 $N,P$     & like & $1$ \\ \hline
 $V,P$     & like & $1$ \\ \hline
 $P,D$     & an & $1$ \\ \hline
 $D,N$     & time & $\frac{2}{5}$ \\
 $D,N$     & arrow & $\frac{2}{5}$ \\
 $D,N$     & flies & $\frac{1}{5}$ \\
 \hline
\end{tabular}
   \end{soln}
  }

\smallskip

  \ex{Based on your $hmm$ constructed in \ref{tri2}. what is the 
  state sequence with the highest probability for the following observation sequence:
  
  \begin{quote}
  \texttt{bos} \texttt{bos} time flies like an arrow \texttt{eos}
  \end{quote}
  
  \begin{soln}
  
  \begin{tabular}{|lllllll|l|}
  \hline
  bos & time & flies & like & an & arrow & eos & \\ \hline
  (bos,bos) & (bos,N) & (N,V) & (V,P) & (P,D) & (D,N) & (N,eos) & \\
   $1$ & $\times 1 \times \frac{2}{5}$ & $\times \frac{1}{2} \times \frac{1}{2}$ & 
   $\times \frac{1}{3} \times 1$ & $\times 1 \times 1$ & 
   $\times 1 \times \frac{2}{5}$ & $\times 1 \times 1$ & $=\frac{1}{75} \ast$ \\ \hline
  (bos,bos) & (bos,N) & (N,V) & (V,V) & (V,D) & (D,N) & (N,eos) & \\
   $1$ & $\times 1 \times \frac{2}{5}$ & $\times \frac{1}{2} \times \frac{1}{2}$ & 
   $\times \frac{1}{3} \times \frac{1}{2}$ & $\times 1 \times 1$ & 
   $\times 1 \times \frac{2}{5}$ & $\times 1 \times 1$ & $=\frac{1}{150}$ \\ \hline
  (bos,bos) & (bos,N) & (N,N) & (N,P) & (P,D) & (D,N) & (N,eos) & \\
   $1$ & $\times 1 \times \frac{2}{5}$ & $\times \frac{1}{2} \times \frac{1}{5}$ & 
   $\times \frac{1}{2} \times 1$ & $\times 1 \times 1$ & 
   $\times 1 \times \frac{2}{5}$ & $\times 1 \times 1$ & $=\frac{1}{125}$ \\ \hline
  (bos,bos) & (bos,N) & (N,N) & (N,V) & (V,D) & (D,N) & (N,eos) & \\
   $1$ & $\times 1 \times \frac{2}{5}$ & $\times \frac{1}{2} \times \frac{1}{5}$ & 
   $\times \frac{1}{2} \times \frac{1}{2}$ & $\times \frac{1}{3} \times 1$ & 
   $\times 1 \times \frac{2}{5}$ & $\times 1 \times 1$ & $=\frac{1}{750}$\\ \hline
  \end{tabular}
  \end{soln}
  }
  
\end{xlist}

\bigskip

\ex\label{trigram} Part-of-speech Tagging:

Consider the task of assigning the most
likely part of speech tag to each word in an input sentence. We want
to get the best (or most likely) tag sequence as defined by the
equation:

\[ T^\ast = \arg\max_{t_0, \ldots, t_n} P(t_0, \ldots, t_n \mid w_0,
\ldots, w_n) \]

\begin{xlist}

{\ex 
Write down the equation for computing the probability $P(t_0, \ldots,
t_n \mid w_0, \ldots, w_n)$ using Bayes Rule and a 
trigram probability model over part of speech tags.

\begin{soln}
 \begin{eqnarray}
 P(t_0, \ldots, t_n \mid w_0, \ldots, w_n) &=& P(w_0, \ldots, w_n \mid
 t_0, \ldots, t_n) \times P(t_0, \ldots, t_n) \nonumber \\
 &=& \left( \prod_{i=0}^n P(w_i \mid t_i) \right) \times
      \left( P(t_0) \times P(t_1 \mid t_0) \times
        \prod_{i=2}^n P(t_i \mid t_{i-2}, t_{i-1}) \right)
      \nonumber
 \end{eqnarray}
\end{soln}
}

\bigskip

{\ex 
%Consider the task of assigning the most
%likely part of speech tag to each word in an input sentence. In standard
%trigram tagging we predict the current tag based on the previous two
%tags: $P(t_i \mid t_{i-2}, t_{i-1})$. 
We realize that we can get better tagging accuracy if we can condition the current tag on
the previous tag and the next tag, i.e. if we can use $P(t_i \mid t_{i-1}, t_{i+1})$. 
Thus, we define the best (or most likely) tag sequence 
as follows:
\begin{eqnarray*}
T^\ast &=& \arg\max_{t_0, \ldots, t_n} P(t_0, \ldots, t_n \mid w_0,
\ldots, w_n) \\
&\approx& \arg\max_{t_0, \ldots, t_n} \prod_{i=0}^{n+1} P(w_i \mid t_i) \times P(t_i \mid t_{i-1}, t_{i+1})
\textrm{ where $t_{-1} = t_{n+1} = $ none }
\end{eqnarray*}
Explain why the Viterbi algorithm cannot be directly used to find $T^\ast$ for the above equation.

\begin{soln}
The standard way of using Viterbi would be to have states in the HMM as pairs of tags
and then we store the best path upto each tag pair 
$\langle t_{i-1}, t_i \rangle$ for time
step $i$ and then we can efficiently compute the best score upto state $t_{i+1}$ recursively as
follows: 
\[ \textrm{Viterbi}[i+1, \langle t_{i+1}, t_i \rangle] = \max_{\langle t_{i-1}, t_i \rangle} \left(
\textrm{Viterbi}[i, \langle t_{i-1}, t_{i} \rangle] \times P(w_{i+1} \mid t_{i+1}) \times P(t_{i+1} \mid \langle t_{i-1}, t_i \rangle) \right) \]

For $t_i$ we can only condition on the previous tag $t_{i-1}$ in the Viterbi algorithm 
since we have entered the score for each possible $t_{i-1}$ in the Viterbi matrix. 
Crucially Viterbi proceeds from left to right, and so it cannot condition on a
tag $t_{i+1}$ which occurs on the right (and is not still part of the Viterbi table). 
Running Viterbi from right to left also does not work, 
due to tag $t_{i-1}$ which occurs on the left. As a result, we cannot simultaneously consider 
$t_{i-1}$ and $t_{i+1}$ for Viterbi computation for the trigram probability
$P(t_i \mid t_{i-1}, t_{i+1})$.

\end{soln}
}

\bigskip

{\ex 
BestScore is the score for the maximum probability tag sequence for a given input word sequence. 
\[ \textrm{BestScore} = \max\limits_{t_0, \ldots, t_n} P(t_0, \ldots, t_n \mid w_0,
\ldots, w_n) \]
It
is a bit simpler to compute than Viterbi since it does not
compute the best sequence of tags (no back pointer is required). For the standard
trigram model $P(t_i \mid  t_{i-2}, t_{i-1})$:

\[ \textrm{BestScore} = \max\limits_{t_0, \ldots, t_n} \prod_{i=0}^{n+1} P(w_i \mid t_i) \times P(t_i \mid  t_{i-2}, t_{i-1}) \]

Assuming that $t_{-1} = t_{-2} = t_{n+1} = $ none, we can compute BestScore recursively from left to right as follows:
\begin{eqnarray*}
\textrm{BestScore}[i+1, t_{i+1}, t_i] & = & \max_{t_{i-1}, t_i} \left(\ 
\textrm{BestScore}[i, t_{i-1}, t_{i}] \times P(w_{i+1} \mid t_{i+1}) \times P(t_{i+1} \mid t_{i-1}, t_i) \ \right) \\
&& \textrm{ for all $-1 \leq i \leq n$ } \\
\textrm{BestScore} &=& \max\limits_{\langle t, \textrm{none} \rangle} \textrm{BestScore}[n+1, \langle t, \textrm{none} \rangle] 
\end{eqnarray*}

This algorithm for computing BestScore is simply the recursive forward algorithm for HMMs but with the sum replaced by max. 

Provide an algorithm 
 in order to compute BestScore for the improved trigram model $P(t_i \mid t_{i-1}, t_{i+1})$:
 \[ \textrm{BestScore} = \max\limits_{t_0, \ldots, t_n} \prod_{i=0}^{n+1} P(w_i \mid t_i) \times P(t_i \mid t_{i-1}, t_{i+1})
\textrm{ where $t_{-1} = t_{n+1} = $ none } \]
As before assume
that:
\( P(t_0 \mid t_{-1} = \textrm{none}, t_1) \approx P(t_0 \mid t_1) \textrm{ and } P(t_n \mid t_{n-1}, t_{n+1} = \textrm{none}) \approx P(t_n \mid t_{n-1}) \)

\bigskip

You can provide either pseudo code, a recursive definition of the algorithm, or
a recurrence relation.

\bigskip

{\em Hint}: The first step would be to extend the recursive BestScore algorithm given above 
to read the input from right to left.

\begin{soln}
We can define 
a dynamic programming algorithm that can be used to efficiently
compute the best sequence of tags for the given equation. The algorithm is shown below as a recursive
function (each call to this function can be stored in a matrix to
give a polynomial time algorithm). We assume the input is padded with none's so
that we can get probabilities
$P(t_0 \mid t_{-1} = \textrm{none}, t_1)$ and $P(t_n \mid t_{n-1}, t_{n+1} = \textrm{none})$.

\begin{tabbing}
123\=456\=789\=012 \kill \\
function bestScore() \\
\> return NotViterbi($n+2$, none, none, none) \\
\\
function NotViterbi($i, t_{i-2}, t_{i-1}, t_i$) \\
\> if ($i-1 == -1$) \\
\> \> if ($t_{i-2}, t_{i-1}, t_i$ == none, none, none) return 1 \\
\> \> else return 0 \\
\> return $\max_{t_{i-1}}$ NotViterbi($i-1, t_{i-3}, t_{i-2}, t_{i-1}$) $\times\ P(w_{i-1} \mid t_{i-1}) \times P(t_{i-1} \mid t_{i-2}, t_i)$
\end{tabbing}
\end{soln}
}

\end{xlist}

\end{exe}
\end{document}

