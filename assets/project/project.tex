%
% File acl2012.tex
%
% Contact: Maggie Li (cswjli@comp.polyu.edu.hk), Michael White (mwhite@ling.osu.edu)
%%
%% Based on the style files for ACL2008 by Joakim Nivre and Noah Smith
%% and that of ACL2010 by Jing-Shin Chang and Philipp Koehn


\documentclass[11pt,letterpaper]{article}
\usepackage[letterpaper]{geometry}
\usepackage{acl2012}
\usepackage{times}
\usepackage{latexsym}
\usepackage{amsmath}
\usepackage{multirow}
\usepackage{url}
\makeatletter
\newcommand{\@BIBLABEL}{\@emptybiblabel}
\newcommand{\@emptybiblabel}[1]{}
\makeatother
\usepackage[hidelinks]{hyperref}
\DeclareMathOperator*{\argmax}{arg\,max}
\setlength\titlebox{6.5cm}    % Expanding the titlebox

\title{Instructions for Project Submissions}

\author{First Author \\
  Affiliation / Address line 1 \\
  Affiliation / Address line 2 \\
  Affiliation / Address line 3 \\
  {\tt email@domain} \\\And
  Second Author \\
  Affiliation / Address line 1 \\
  Affiliation / Address line 2 \\
  Affiliation / Address line 3 \\
  {\tt email@domain} \\}

\date{}

\begin{document}
\maketitle
\begin{abstract}
  This document contains the instructions for preparing a manuscript for submission for the NLP class project. The document itself conforms to its own specifications, and is therefore an example of what your manuscript should look like. These instructions should be used for both papers submitted for review and for final versions of accepted papers. Authors are asked to conform to all the directions reported in this document.
\end{abstract}

\section{Credits}

This document has been adapted from the instructions for ACL proceedings, including those for ACL-2008 by Johanna D. Moore, Simone Teufel, James Allan, and Sadaoki Furui, those for ACL-2005 by Hwee Tou Ng and Kemal Oflazer, those for ACL-2002 by Eugene Charniak and Dekang Lin, and earlier ACL and EACL formats. Those versions were written by several people, including John Chen, Henry S. Thompson and Donald Walker. Additional elements were taken from the formatting instructions of the {\em International Joint Conference on Artificial Intelligence}.

\section{General Instructions}

Manuscripts must be in two-column format. Exceptions to the two-column format include the title,
authors' names and complete addresses, which must be centered at the top of the first page,
and any full-width figures or tables (see the guidelines in Subsection~\ref{ssec:first}). {\bf Type single-spaced}.
Start all pages directly under the top margin. See the guide-lines later regarding formatting the first page. Do not number the pages.

\subsection{Format of Electronic Manuscript}
\label{sect:pdf}

For the production of the electronic manuscript you must use Adobe's
Portable Document Format (PDF). This format can be generated from
postscript files: on Linux/Unix systems, you can use {\tt ps2pdf} for this
purpose; under Microsoft Windows, you can use Adobe's Distiller, or
if you have {\tt cygwin} installed, you can use {\tt dvipdf} or
{\tt ps2pdf}.  Note
that some word processing programs generate PDF which may not include
all the necessary fonts (esp. tree diagrams, symbols). When you print
or create the PDF file, there is usually an option in your printer
setup to include none, all or just non-standard fonts.  Please make
sure that you select the option of including ALL the fonts.  {\em Before sending it, test your PDF by printing it from a computer different from the one where it was created}. Moreover,
some word processor may generate very large postscript/PDF files,
where each page is rendered as an image. Such images may reproduce
poorly.  In this case, try alternative ways to obtain the postscript
and/or PDF.  One way on some systems is to install a driver for a
postscript printer, send your document to the printer specifying
``Output to a file'', then convert the file to PDF.

Additionally, it is of utmost importance to specify the {\bf US-Letter format} (8.5in $\times$ 11in) when formatting the paper. When working with {\tt dvips}, for instance, one should specify {\tt -t letter}.

Print-outs of the PDF file on US-Letter paper should be identical to the
hardcopy version.  If you cannot meet the above requirements about the
production of your electronic submission, please contact the
publication chair above as soon as possible.


\subsection{Layout}
\label{ssec:layout}

Format manuscripts two columns to a page, in the manner these
instructions are formatted. The exact dimensions for a page on US-letter
paper are:

\begin{itemize}
\item Left and right margins: 1in
\item Top margin:1in
\item Bottom margin: 1in
\item Column width: 3.15in
\item Column height: 9in
\item Gap between columns: 0.2in
\end{itemize}

\noindent Papers should not be submitted on any other paper size. If you cannot meet the above requirements about the production of your electronic submission, please contact the publication chair above as soon as possible.

\subsection{Fonts}

For reasons of uniformity, Adobe's {\bf Times Roman} font should be
used. In \LaTeX2e{} this is accomplished by putting

\begin{quote}
\begin{verbatim}
\usepackage{times}
\usepackage{latexsym}
\end{verbatim}
\end{quote}
in the preamble. If Times Roman is unavailable, use {\bf Computer
  Modern Roman} (\LaTeX2e{}'s default).  Note that the latter is about
  10\% less dense than Adobe's Times Roman font.


\begin{table}[h]
\begin{center}
\begin{tabular}{|l|rl|}
\hline \bf Type of Text & \bf Font Size & \bf Style \\ \hline
paper title & 15 pt & bold \\
author names & 12 pt & bold \\
author affiliation & 12 pt & \\
the word ``Abstract'' & 12 pt & bold \\
section titles & 12 pt & bold \\
document text & 11 pt  &\\
captions & 10 pt & \\
abstract text & 10 pt & \\
bibliography & 10 pt & \\
footnotes & 9 pt & \\
\hline
\end{tabular}
\end{center}
\caption{\label{font-table} Font guide. }
\end{table}

\subsection{The First Page}
\label{ssec:first}

Center the title, author's name(s) and affiliation(s) across both
columns. Do not use footnotes for affiliations.  Do not include the
paper ID number assigned during the submission process.
Use the two-column format only when you begin the abstract.

{\bf Title}: Place the title centered at the top of the first page, in
a 15 point bold font.  (For a complete guide to font sizes and styles, see Table~\ref{font-table}.)
Long title should be typed on two lines without
a blank line intervening. Approximately, put the title at 1in from the
top of the page, followed by a blank line, then the author's names(s),
and the affiliation on the following line.  Do not use only initials
for given names (middle initials are allowed). Do not format surnames
in all capitals (e.g., ``Zhou,'' not ``ZHOU'').  The affiliation should
contain the author's complete address, and if possible an electronic
mail address. Leave about 0.75in between the affiliation and the body
of the first page. The title, author names and addresses should be completely identical to those entered to the electronic paper submission website in order to maintain the consistency of author information among all publications of the conference.

{\bf Abstract}: Type the abstract at the beginning of the first
column.  The width of the abstract text should be smaller than the
width of the columns for the text in the body of the paper by about
0.25in on each side.  Center the word {\bf Abstract} in a 12 point
bold font above the body of the abstract. The abstract should be a
concise summary of the general thesis and conclusions of the paper.
It should be no longer than 200 words. The abstract text should be in 10 point font.

{\bf Text}: Begin typing the main body of the text immediately after
the abstract, observing the two-column format as shown in
the present document. Do not include page numbers.

{\bf Indent} when starting a new paragraph. For reasons of uniformity,
use Adobe's {\bf Times Roman} fonts, with 11 points for text and
subsection headings, 12 points for section headings and 15 points for
the title.  If Times Roman is unavailable, use {\bf Computer Modern
  Roman} (\LaTeX2e's default; see section \ref{sect:pdf} above).
Note that the latter is about 10\% less dense than Adobe's Times Roman
font.

\subsection{Sections}

{\bf Headings}: Type and label section and subsection headings in the
style shown on the present document.  Use numbered sections (Arabic
numerals) in order to facilitate cross references. Number subsections
with the section number and the subsection number separated by a dot,
in Arabic numerals. Do not number subsubsections.

{\bf Citations}: Citations within the text appear
in parentheses as~\cite{Gusfield:97} or, if the author's name appears in
the text itself, as Gusfield~\shortcite{Gusfield:97}. Append lowercase letters to the year in cases of ambiguities. Treat double authors as in~\cite{Aho:72}, but write as in~\cite{Chandra:81} when more than two authors are involved. Collapse multiple citations as in~\cite{Gusfield:97,Aho:72}. Also refrain from using full citations as sentence constituents. We suggest that instead of
\begin{quote}
  ``\cite{Gusfield:97} showed that ...''
\end{quote}
you use
\begin{quote}
``Gusfield \shortcite{Gusfield:97}   showed that ...''
\end{quote}

If you are using the provided \LaTeX{} and Bib\TeX{} style files, you
can use the command \verb|\newcite| to get ``author (year)'' citations.

As reviewing will be double-blind (except that action editors know author identity and authors know action-editor identity), the submitted version of the papers should not include the
authors' names and affiliations. Furthermore, self-references that
reveal the author's identity, e.g.,
\begin{quote}
``We previously showed \cite{Gusfield:97} ...''
\end{quote}
should be avoided. Instead, use citations such as
\begin{quote}
``Gusfield \shortcite{Gusfield:97}
previously showed ... ''
\end{quote}

To be clear: You should reference your prior work if it is relevant; but use the third person instead of the 1st person and in place of references like “(Anonymous) showed…”, since such anonymized references do not allow readers to examine relevant related work.

Authors’ names should also be removed from the ``Document Properties'' display that can be viewed using Adobe Acrobat’s ``File $\rightarrow$ Properties'' menu.

Please do not include acknowledgements when submitting your papers. Papers that do not conform
to these requirements may be rejected without review.

\textbf{References}: Gather the full set of references together under
the heading {\bf References}; place the section before any Appendices,
unless they contain references. Arrange the references alphabetically
by first author, rather than by order of occurrence in the text.
Provide as complete a citation as possible, using a consistent format,
such as the one for {\em Computational Linguistics\/} or the one in the
{\em Publication Manual of the American
Psychological Association\/}~\cite{APA:83}.  Use of full names for
authors rather than initials is preferred.  

The \LaTeX{} and Bib\TeX{} style files provided roughly fit the
American Psychological Association format, allowing regular citations,
short citations and multiple citations as described above.

{\bf Appendices}: Appendices, if any, directly follow the text and the
references (but see above).  Letter them in sequence and provide an
informative title: {\bf Appendix A. Title of Appendix}.

\textbf{Acknowledgment} sections should go as a last section immediately
before the references. Do not number the acknowledgement section.

\subsection{Footnotes}

{\bf Footnotes}: Put footnotes at the bottom of the page. They may
be numbered or referred to by asterisks or other
symbols.\footnote{This is how a footnote should appear.} Footnotes
should be separated from the text by a line.\footnote{Note the
line separating the footnotes from the text.}  Footnotes should be in 9 point font.

\subsection{Graphics}

{\bf Illustrations}: Place figures, tables, and photographs in the
paper near where they are first discussed, rather than at the end, if
possible.  Wide illustrations may run across both columns and should be placed at
the top of a page. Color illustrations are discouraged, unless you have verified that
they will be understandable when printed in black ink.

{\bf Captions}: Provide a caption for every illustration; number each one
sequentially in the form:  ``Figure 1. Caption of the Figure.'' ``Table 1.
Caption of the Table.''  Type the captions of the figures and
tables below the body, using 10 point text.

\section{Translation of non-English Terms}

It is also advised to supplement non-English characters and terms
with appropriate transliterations and/or translations
since not all readers understand all such characters and terms.

Inline transliteration or translation can be represented in
the order of: original-form transliteration ``translation''.

\section{Length of Submission}
\label{sec:length}

Submissions may consist of four to seven (4-7) letter format (not A4) pages of content and unlimited additional pages (only) allowed for references. Papers that are revisions of submissions with prior (b) or (c) decisions may be allowed one to two additional pages of content to accommodate required revisions.

Appendices (if any) are counted as content pages. Papers that do not conform to the specified length and
formatting requirements are subject to re-submission.


\section*{Acknowledgments}

Do not number the acknowledgment section. Do not include this section when submitting your paper for review.
\bibliography{project}
\bibliographystyle{acl2012}

\end{document}



